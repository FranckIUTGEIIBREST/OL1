Pour cet exercice, le logiciel va vous aider � �noncer des conjectures sur une fonction � valeurs complexes. L'exercice demande un travail de r�flexion et une �tude "sur papier".
\section{�criture d'une fonction en {\fontfamily{lmtt}\selectfont Python}}

On souhaite d�finir la fonction $\displaystyle f(t)=\frac{1}{1+t^2}$ et calculer des valeurs de cette fonction en certains points. Pour l'exemple, on a choisi les valeurs suivantes pour la variable $t$ : 0, 1, e et $\pi$. Il est donc difficile ici d'utiliser la commande {\fontfamily{lmtt}\selectfont arange}

Il vous est demand� de lire et comprendre le code ci-dessous :
\begin{lstlisting}[language={Python}, basicstyle=\ttfamily \footnotesize, showstringspaces=false, keywordstyle=\color{blue}, commentstyle=\color{vert}, stringstyle=\color{red}, identifierstyle=\ttfamily,frame=trBL] 
from numpy import *

#Attention l'indentation est PRIMORDIALE !!!!

def f(t):
   import numpy as np  # non necessaire car deja importee
   
   return 1/(1+t**2)  # 1/(1+t^2)
#fin de la fonction

abscisse=array([0,1,exp(1),pi])
valeurEchantillon=f(abscisse)

print(valeurEchantillon) 
\end{lstlisting}
\section{D�finition d'une fonction complexe $H(\omega)$ dite fonction de transfert}
Soit $\omega_0$ un nombre r�el strictement positif et soit $h$ la fonction r�elle de la variable $p$ d�finie par $\displaystyle h(p)=\frac{\sqrt{3}}{1+\frac{p}{\omega_0}}$

On d�finit la fonction $H$, dans $\mathbb{R}^+$, par $\displaystyle H(\omega)=h(j\omega)=\frac{\sqrt{3}}{1+j\frac{\omega}{\omega_0}}$
\begin{enumerate}
 \item Donner l'expression exacte en fonction de $\omega$ de
\bigskip
  \begin{itemize}
   \item[$\bullet$] la partie r�elle de $\displaystyle H(\omega)$ : $\mathrm{Re}\left(H(\omega)\right)=$
\bigskip
    \item[$\bullet$] la partie imaginaire de $\displaystyle H(\omega)$ : $\mathrm{Im}\left(H(\omega)\right)=$
\bigskip
    \item[$\bullet$] le module de $\displaystyle H(\omega)$: $\left|H(\omega)\right|=$
\bigskip
    \item[$\bullet$] l'argument de $\displaystyle H(\omega)$ : $\mathrm{Arg}\left(H(\omega)\right)=$        
  \end{itemize}
\bigskip
\item Donner un algorithme sur papier qui renvoit l'argument d'un nombre complexe $u=a+jb$.\\
\underline{Remarques:} 
 \begin{itemize}
   \item[$\bullet$] le pr�c�dent TP a fourni les fonctions pour obtenir la partie r�elle et la partie imaginaire d'un nombre complexe
   \item[$\bullet$] {\fontfamily{lmtt}\selectfont arctan(x)} est la fonction arctangente qui renvoit le r�el $\theta$ de l'intervalle $\displaystyle \left]- \frac{\pi}{2}; \frac{\pi}{2}\right[$ tel que $\displaystyle \tan(\theta)=x$
  \end{itemize}
\item Premiers calculs: calculer la partie r�elle, la partie imaginaire, le module et l'argument de $\displaystyle H(\omega)$ lorsque $\omega=0$, puis lorsque $\omega=\omega_0$.
\bigskip
 \item Remplir le tableau ci-apr�s en vous servant du logiciel \texttt{Python}.\\
  
 %\textbf{Rappel :} pour calculer le module et l'argument d'un nombre complexe sous \texttt{Scilab}, on utilise la commande \texttt{polar}. Par exemple, si \texttt{u} est un nombre complexe, {\fontfamily{lmtt}\selectfont [r, theta]=polar(u)} calcule le module et l'argument du \texttt{u}.

 \renewcommand{\arraystretch}{2}
 \begin{tabular}{|p{2.8cm}|p{1.5cm}|p{2.8cm}|p{2.8cm}|} \hline
 \centering Nombre complexe & \centering $\displaystyle \frac{\omega}{\omega_0}$ & \centering Module &  Argument \\ \hline
 \centering$\displaystyle H(0)$ & & &\\ \hline
 \centering$\displaystyle H\left(\frac{\omega_0}{100}\right)$ & & &\\ \hline
 \centering$\displaystyle H\left(\frac{\omega_0}{10}\right)$ & & &\\ \hline
 \centering$\displaystyle H\left(\frac{\omega_0}{3}\right)$ & & &\\ \hline
 \centering$\displaystyle H\left(\frac{\omega_0}{2}\right)$ & & &\\ \hline
 \centering$\displaystyle H(\omega_0)$ & & &\\ \hline
 \centering$\displaystyle H(2\omega_0)$ & & &\\ \hline
 \centering$\displaystyle H(3\omega_0)$ & & &\\ \hline
 \centering$\displaystyle H(10\omega_0)$ & & &\\ \hline
 \centering$\displaystyle H(100\omega_0)$ & & &\\ \hline
\end{tabular} 
\renewcommand{\arraystretch}{1}
 \bigskip\\
 

\end{enumerate}
\begin{corrige}
\begin{lstlisting}[language={Python}, basicstyle=\ttfamily \footnotesize, showstringspaces=false, keywordstyle=\color{blue}, commentstyle=\color{vert}, stringstyle=\color{red}, identifierstyle=\ttfamily,frame=trBL] 
from numpy import *
from matplotlib.pyplot import *

w0=3141

def module(w):
   import numpy as np  # non necessaire
   #import matplotlib.pyplot as plt  # non necessaire

   return sqrt(3)/sqrt(1+(w/w0)**2)  # 1/(1+t^2)  # 1/(1+t^2)

def argum(w):
   import numpy as np  # non necessaire
   #import matplotlib.pyplot as plt  # non necessaire
   
   return(-arctan(w/w0))


omega=array([0,w0/100,w0/10,w0/3,w0/2,w0,2*w0,3*w0,10*w0,100*w0])
colonne_2=module(omega)
print(colonne_2)

colonne_3=argum(omega)
print(colonne_3)
\end{lstlisting}
\end{corrige}
\section{�tude sommaire du module $\left| H(\omega)\right|$.} D�terminer les limites suivantes :
  \begin{enumerate}
   \item $\displaystyle \lim_{\omega \rightarrow 0^{+}}\left| H(\omega)\right|$
   \item $\displaystyle \lim_{\omega \rightarrow \omega_0}\left| H(\omega)\right|$
   \item $\displaystyle \lim_{\omega \rightarrow +\infty}\left| H(\omega)\right|$
   \item Esquisser sur papier l'allure de la courbe repr�sentative $\displaystyle \omega \mapsto \left| H(\omega)\right|$.
   \item Proposer un mod�le tr�s simple, voire simpliste, pour la courbe repr�sentative de la fonction $\displaystyle \omega \mapsto \left| H(\omega)\right|$
  \end{enumerate} 
\section{�tude sommaire de l'argument $\mathrm{arg}\left( H(\omega)\right)$.} D�terminer les limites suivantes :
  \begin{enumerate}
   \item $\displaystyle \lim_{\omega \rightarrow 0^{+}}\mathrm{arg}\left(H(\omega)\right)$
   \item $\displaystyle \lim_{\omega \rightarrow \omega_0}\mathrm{arg}\left(H(\omega)\right)$
   \item $\displaystyle \lim_{\omega \rightarrow +\infty}\mathrm{arg}\left(H(\omega)\right)$
   \item Esquisser sur papier l'allure de la courbe repr�sentative $\displaystyle \omega \mapsto \mathrm{arg}\left(H(\omega)\right)$.
   \item Proposer un mod�le tr�s simple, voire simpliste, pour la courbe repr�sentative de la fonction $\displaystyle \omega \mapsto \mathrm{arg}\left(H(\omega)\right)$
  \end{enumerate} 
\bigskip
