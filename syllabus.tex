\begin{center}
\Large{Syllabus}
\end{center}
Le programme p�dagogique national du dipl�me universitaire de technologie sp�cialit� G�nie �lectrique et Informatique Industrielle comprend, en premi�re ann�e, deux modules intitul�s "Outils logiciels". 
Les premiers Travaux Pratiques vont vous permettre de vous familiariser avec le langage Python, en utilisant l'environnement de d�veloppement int�gr� \texttt{Spyder}. Vous pouvez t�l�charger gratuitement le logiciel \texttt{Spyder} � l'adresse
\begin{center}
\textbf{http://pythonhosted.org/spyder/installation.html}
\end{center}
\bigskip
Les th�mes abord�s durant l'ann�e seront :
\begin{enumerate}
\item Les repr�sentations graphiques
\item Les fonctions sinuso�dales
\item Les diagrammes de Bode
\item Les asymptotes
\item Les polyn�mes
\item La d�composition en �l�ments simples
\item L'int�gration
\item Les s�ries de Fourier
\end{enumerate}
