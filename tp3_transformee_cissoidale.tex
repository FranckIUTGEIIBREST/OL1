\textbf{Somme de 2 fonctions sinuso�dales de m\^eme fr�quence}

Probl�me: Si  $$u_1(t)=A_1\cos(\omega t+\varphi_1)$$ et $$u_2(t)=A_2\cos(\omega t+\varphi_2)$$ on veut d�terminer $A$ et $\varphi$ tels que $$u_1(t)+u_2(t)=A\cos(\omega t+\varphi)$$
$i.e$ tels que
$$A_1\cos(\omega t+\varphi_1)+A_2\cos(\omega t+\varphi_2)=A\cos(\omega t+\varphi)$$

\section{Observations graphiques}
Recopier et ex�cuter le script suivant.
\begin{lstlisting}[language={Python}, basicstyle=\ttfamily \footnotesize, showstringspaces=false, keywordstyle=\color{blue}, commentstyle=\color{vert}, stringstyle=\color{red}, identifierstyle=\ttfamily,frame=trBL] 
from matplotlib.pyplot import * 
from numpy import *

x=arange(-2*pi,2*pi,.05)

y1=cos(pi*x)
y2=2*cos(pi*x+pi/3)

clf()
plot(x,y1,'-g')
plot(x,y2,'-r')
plot(x,y1+y2,'-k')

axis([-2*pi,2*pi,-3,3])
grid() #on ajoute une grille
show()
\end{lstlisting}


\begin{enumerate}
\item Quelles sont les valeurs de $A_1$, $A_2$, $\varphi_1$ et $\varphi_2$?
%\item Que pensez-vous de la fonction $u$?
\item Estimer graphiquement $A$ et $\varphi$.
\item Rajouter une ligne de commande dans le script pour tracer en noir $A \cos (\pi t+\varphi)$ avec les param�tres $A$ et $\varphi$ que vous aurez estim�s.
\item Faire varier $A_2$ et $\varphi_2$ et observer comment varient $A$ et $\varphi$.
\item Les variations de $A$ d�pendent-elles de $A_1$ et $A_2$? de $\varphi_1$ et $\varphi_2$?
\item M�mes questions pour les variations de $\varphi$.
\end{enumerate}


\section{Plongement complexe: transform�e cisso�dale}
On va "plonger" le probl�me r�el dans le domaine des nombres complexes o� sa r�solution est plus simple.
\subsection{Notation complexe d'une onde sinuso�dale}
A toute fonction $u(t)=A\cos(\omega t+\varphi)$, on peut lui associer la fonction complexe $\underline{U(t)}$:
\begin{center}
 $\begin{array}{lcl}
   \underline{U(t)} & =& A\cos(\omega t+\varphi)+jA\sin(\omega t+\varphi) \\
   & = & A(\cos(\omega t+\varphi)+j\sin(\omega t+\varphi)) \\
   & = & A\mathrm{e}^{j(\omega t+\varphi)} \\
   & = & A\mathrm{e}^{j\omega t}\mathrm{e}^{j\varphi} \\
 \end{array}$
\end{center}

$u(t)$ est la partie r�elle de  $\underline{U(t)}$\\
On a donc
\begin{center}
\begin{tabular}{|c|}  \hline
\\
$\underline{U(t)}=A\mathrm{e}^{j(\omega t+\varphi)}=A\mathrm{e}^{j\omega t}\mathrm{e}^{j\varphi}$\\
\\
\hline
\end{tabular} 
\end{center}

\subsection{Somme de deux ondes sinuso�dales de m�me pulsation}
Soit 

$u_1(t)=A_1\cos(\omega t+\varphi_1)\qquad$  D�terminer $\underline{U_1(t)}$
\\

$u_2(t)=A_2\cos(\omega t+\varphi_2)\qquad$  D�terminer $\underline{U_2(t)}$
\bigskip
\begin{enumerate}
 \item Si $u(t)=u_1(t)+u_2(t)$, d�terminer $\underline{U(t)}$ en fonction de $\underline{U_1(t)}$ et $\underline{U_2(t)}$.
 \item Exprimez $\underline{U_1(t)}$, $\underline{U_2(t)}$ puis $\underline{U(t)}$ en fonction de  $A_1$, $A_2$, $\varphi_1$ et $\varphi_2$.
 \item D�montrer que $A$ et $\varphi$ v�rifient $A\mathrm{e}^{j\varphi}=A_1\mathrm{e}^{j\varphi_1}+A_2\mathrm{e}^{j\varphi_2}$
\end{enumerate}

\subsection{Application}
Soit $u_1(t)=\cos\left(\pi t+\displaystyle \frac{\pi}{2}\right)$ et $u_2(t)=\cos\left(\pi t-\displaystyle \frac{\pi}{6}\right)$.
\begin{enumerate}
\item Ecrire $\underline{U_1(t)}$ et $\underline{U_2(t)}$.
\item Calculer $\underline{U(t)}$ puis d�terminer $A$ et $\varphi$.
\item V�rifier graphiquement la concordance de vos r�sultats. 
\end{enumerate}
\section{Les complexes avec {\fontfamily{lmtt}\selectfont Python}}
 Le code suivant se trouve dans le fichier  {\fontfamily{lmtt}\selectfont calcul\_complexe.py} dans le dossier {\fontfamily{lmtt}\selectfont docum.geii/PYHTON\_SPYDER\_OL}
\begin{lstlisting}[language={Python}, basicstyle=\ttfamily \footnotesize, showstringspaces=false, keywordstyle=\color{blue}, commentstyle=\color{vert}, stringstyle=\color{red}, identifierstyle=\ttfamily,frame=trBL] 

from cmath import * #on importe les fonctions mathematiques pour les
		    # nombres complexes
		    # elles sont stockees dans la bibliotheque cmath

z=1+1*1j # ou bien z=complex(1,1)
print(z)        #on affiche le complexe z
print(polar(z))  #on affiche le module et l'argument de z

w=-2+3*1j
print(w)  
print('Re(w)=',w.real)   #on affiche la partie reelle de w
print('Im(w)=',w.imag)  #on affiche la partie imaginaire de w
print(w.conjugate()) #on demande d'afficher le conjuge de w

print('|z|=',abs(w))  #on affiche le module du complexe w
print(w**3) #on affiche le calcul de w^3
\end{lstlisting}

Pour obtenir l'aide sur la biblioth�que  {\fontfamily{lmtt}\selectfont cmath} qui contient les fontions math�matiques pour les nombres complexes, taper {\fontfamily{lmtt}\selectfont https://docs.python.org/2/library/cmath.html}
\section{Exercices}

En utilisant les d�phasages et la technique de la partie 3 �crire sous forme $A\cos(\omega t+\varphi)$ les fonctions suivantes:
\begin{enumerate}
\item $\displaystyle f_1(t)=2\sin\left(3t+ \frac{\pi}{3}\right)$
\item $\displaystyle f_2(t)=f_1(t)+2\cos\left(3t-\frac{5\pi}{6}\right)$
\item $\displaystyle f_3(t)=3\cos\left(200\pi t+ \frac{\pi}{3}\right)+4\cos\left(200\pi t+ \frac{\pi}{4}\right)$
\item En choisissant une �chelle adapt�e, v�rifiez vos r�sultats gr�ce � une repr�sentation graphique.
\item \textbf{Faire valider} par l'enseignant.
\end{enumerate}
\bigskip

