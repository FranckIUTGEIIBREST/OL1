\documentclass[french,10pt]{report}
\usepackage{a4}
\usepackage[T1]{fontenc}
%\usepackage[utf8]{inputenc}
\usepackage{babel}
\usepackage{array}
\usepackage{delarray}
\usepackage{listings}
\usepackage{multicol}

\usepackage{amssymb}

%\usepackage[L]{thmbox}
\newtheorem{Def}{D�finition}
\newtheorem{theo}{Th�or�me}
\newtheorem{lemme}{Lemme}
\newtheorem{cor}{Corollaire}
\newtheorem{question}{Question}
\newtheorem{corrige}{Correction}[section]
\usepackage[dvips]{graphics} 

%pour les graphiques
\usepackage{graphicx}
\usepackage{transparent} %transparence d'une image

\usepackage{xcolor}

\usepackage{comment}
\includecomment{comment}
%\excludecomment{comment}

\usepackage{vmargin}
\setmarginsrb{2cm}{1cm}{2cm}{1cm}{1cm}{1cm}{1cm}{1cm}  
%\pagestyle{empty}


\usepackage{tikz,pgfplots,pgf}
\usetikzlibrary{arrows,shapes,arrows,positioning,calc,trees,snakes,plotmarks} 

% En-t�te et pied de page
\usepackage{lastpage}
\usepackage{fancyhdr}
\pagestyle{fancy}
\lhead{}
\chead{}
\rhead{}
\fancyfoot{}
\lfoot{\tiny{OL1 - OL2}}
\cfoot{\tiny{\leftmark}}
\rfoot{\tiny{\thepage /\pageref{LastPage}}}

\renewcommand{\headrulewidth}{0pt} %supprime la ligne pr�sente par d�faut pour l'en-t�te
\headwidth 17cm %longueur de l'en-t�te et pied de page
 
\usepackage{vmargin}
\setmarginsrb{2cm}{1cm}{2cm}{1cm}{1cm}{1cm}{1cm}{1cm} 

% page de garde
\title{\textbf{Outils logiciels - {\fontfamily{lmtt}\selectfont Spyder \& Python}}}
\author{G.Bossard - V.Choqueuse - L.Fagon - F.Le Bolc'h - MA.Tirat}
\date{}

%fond de la page de titre
 \usepackage{eso-pic}
 \newcommand\BackgroundPic{%
    \put(72,120){%
 		\parbox[b][\paperheight]{\paperwidth}{%
 		\vfill 
 		\centering
 		{\transparent{0.5}\includegraphics[width=0.5\paperwidth,height=0.4\paperheight,keepaspectratio]{images/logo_spyder.jpg}}
 %\vfill
 		}
 	}
 }
%fin fond d'�cran 


%listing
\lstset{						% Param�trage pour les listings en C
	language={Scilab}, 
	basicstyle=\ttfamily \footnotesize, 
	showstringspaces=false, 
	keywordstyle=\color{blue}, 
	commentstyle=\color{vert}, 
	stringstyle=\color{red}, 
	identifierstyle=\ttfamily,
	frame=trBL
}
%fin listing 

\begin{document}
\definecolor{vert}{rgb}{0.2,0.6,0.4} 
\AddToShipoutPicture*{\BackgroundPic}
\maketitle
\begin{multicols}{2}
\setlength{\columnseprule}{1pt}
\tableofcontents
\end{multicols}
\newpage
%\bigskip
\input{"syllabus"}
\chapter{Introduction - Repr�sentation graphique}
\input{"tp1_intro_graphique"}
\chapter{Fonctions associ�es}
\input{"tp2_fct_associees"}
\chapter{Transform�e cisso�dale}
\input{"tp3_transformee_cissoidale"}
\chapter{Une fonction complexe}
\input{"tp_bode1"}
\chapter{Trac�s de fonctions avec une �chelle logarithmique}
\input{"tp_bode2"}
%\chapter{Asymptotes}
%\input{"tp5_asymptotes"}
%\chapter{Synth�se}
%\input{"synthese"}
%\chapter{Evaluation}
%\input{"corr_tp_type"}
%\chapter{Evaluation groupe 11A}
%\input{"controle11A"}
%\chapter{Polyn\^omes-Fractions rationnelles}
%\input{"tp_poly_fr"}
%\chapter{S�ries de Fourier}
%\input{"Fourier1"}
%\chapter{Evaluation S2}
%\chapter{Evaluation S2}
%\input{"controle11"}
\end{document}
